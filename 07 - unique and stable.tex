\section{Uniqueness and Stability of the Mass Gap}

We prove that the mass gap solution is unique and stable under small perturbations 
of the theory. This involves demonstrating the uniqueness of the mass 
gap and showing that it remains positive and finite under small perturbations 
of the action or parameters of the theory.

\section{Uniqueness}

\subsection{Spectral Representation and Mass Gap}

The two-point correlation function \(G(x-y)\) for a scalar field \(\phi(x)\) in Euclidean space is given by the spectral representation:
\begin{equation}
G(x-y) = \langle \phi(x) \phi(y) \rangle = \int_0^\infty d\mu^2 \, \rho(\mu^2) e^{-\mu |x-y|},
\end{equation}
where \(\rho(\mu^2)\) is the spectral density function, which is non-negative, \(\rho(\mu^2) \geq 0\), and encodes the mass spectrum of the theory.

\subsection{Lowest Mass State and Uniqueness}

The mass gap \(m_{\text{gap}}\) is defined as the lowest non-zero value of \(\mu\) for which \(\rho(\mu^2) \neq 0\). Since \(\rho(\mu^2)\) is non-negative and normalized, there is a single lowest \(\mu = m_{\text{gap}}\) that contributes to the spectral density.

For \(\mu < m_{\text{gap}}\), \(\rho(\mu^2) = 0\). This implies that the lowest mass state dominates the long-distance behavior of \(G(x-y)\), leading to the exponential decay:
\begin{equation}
G(x-y) \sim e^{-m_{\text{gap}} |x-y|} \quad \text{as} \quad |x-y| \to \infty.
\end{equation}

\subsection{Uniqueness Argument}

If there were two different mass gaps, \(m_1\) and \(m_2\) (\(m_1 < m_2\)), they would both contribute to the long-distance behavior of \(G(x-y)\), leading to:
\begin{equation}
G(x-y) \sim c_1 e^{-m_1 |x-y|} + c_2 e^{-m_2 |x-y|} + \ldots,
\end{equation}
where \(c_1, c_2 > 0\) are coefficients.

Since \(m_1 < m_2\), the term \(e^{-m_1 |x-y|}\) would dominate as \(|x-y| \to \infty\), implying that \(m_1\) is the true mass gap. Therefore, there can only be one such \(m_1 = m_{\text{gap}}\).



\section{Stability}

\subsection{Perturbation of the Action}

Consider a perturbed action \(S_\epsilon[U]\) given by:
\begin{equation}
S_\epsilon[U] = S[U] + \epsilon \Delta S[U],
\end{equation}
where \(\epsilon\) is a small parameter, and \(\Delta S[U]\) is a perturbation term.

\subsection{Effect on Correlation Functions}

The perturbed two-point correlation function \(G_\epsilon(x-y)\) becomes:
\begin{equation}
G_\epsilon(x-y) = \langle \phi(x) \phi(y) \rangle_\epsilon = \frac{\int \mathcal{D}U \, \phi(x) \phi(y) e^{-S_\epsilon[U]}}{\int \mathcal{D}U \, e^{-S_\epsilon[U]}}.
\end{equation}
For small \(\epsilon\), we expand the exponential \(e^{-S_\epsilon[U]} \approx e^{-S[U]} (1 - \epsilon \Delta S[U] + \mathcal{O}(\epsilon^2))\).

\subsection{Perturbative Expansion and Mass Gap}

The leading-order correction to the correlation function due to the perturbation can be written as:
\begin{equation}
G_\epsilon(x-y) = G(x-y) + \epsilon \langle \phi(x) \phi(y) \Delta S[U] \rangle + \mathcal{O}(\epsilon^2).
\end{equation}
Since the original correlation function \(G(x-y)\) decays exponentially with the mass gap \(m_{\text{gap}}\), the correction term does not introduce a slower decay unless it modifies the fundamental structure of the theory.

\subsection{Stability of the Mass Gap}

To alter the mass gap, the perturbation \(\Delta S[U]\) would need to create a new state with a smaller mass than \(m_{\text{gap}}\). For small \(\epsilon\), such a significant change in the spectrum is unlikely because it would require a non-perturbative shift in the entire structure of the field configurations.

Therefore, for sufficiently small \(\epsilon\), the mass gap remains stable, and no new massless or lower-mass states appear.

\subsection{Non-Perturbative Argument}

Even under small non-perturbative changes, such as a shift in the background field configurations, the continuity of the spectral density \(\rho(\mu^2)\) ensures that the mass gap \(m_{\text{gap}}\) does not suddenly change or become zero. Thus, the mass gap is robust against small perturbations of the theory, both perturbative and non-perturbative.



\section{Implications}

We have shown that the mass gap solution is unique and stable under small perturbations of the theory. The uniqueness follows from the spectral representation and the fact that the lowest mass state dominates the long-distance behavior of the correlation functions. The stability follows from the perturbative and non-perturbative arguments that small changes in the action do not significantly alter the mass gap or lead to new massless states.
