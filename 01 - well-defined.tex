\section{Existence of Continuum Limit and Persistence of Mass Gap }

We prove the existence of a well-defined continuum limit as the lattice spacing 
\(a \to 0\) for entropic lattice gauge theory. Furthermore, we show that the mass 
gap persists in this limit and is independent of the regularization scheme.

\section{Existence of the Continuum Limit}

\subsection{Scaling and Continuum Extrapolation}

The lattice theory is formulated with a lattice spacing \(a\), introducing a natural ultraviolet (UV) cutoff of order \(1/a\). To take the continuum limit, we consider a sequence of lattice theories with decreasing \(a\), aiming to recover the continuum theory as \(a \to 0\).

\subsection{Correlation Functions and Continuum Limit}

Consider a generic correlation function on the lattice, \(G(x, y; a)\), which depends on the lattice spacing \(a\). The continuum limit of this correlation function is defined as:
\begin{equation}
G_{\text{cont}}(x, y) = \lim_{a \to 0} G(x, y; a),
\end{equation}
where \(G_{\text{cont}}(x, y)\) is the correlation function in the continuum theory.

\subsection{Renormalization and Scaling Behavior}

To ensure a well-defined continuum limit, we perform a renormalization procedure, adjusting parameters as a function of \(a\) to keep physical observables finite. The renormalization group (RG) flow is governed by the beta function \(\beta(g)\):
\begin{equation}
\frac{dg(a)}{d\log a} = \beta(g(a)).
\end{equation}
The continuum limit is achieved when \(g(a)\) approaches a fixed point \(g^*\) such that \(\beta(g^*) = 0\).



\section{Renormalization and Scaling Behavior}

\subsection{Lattice Renormalization}

On the lattice, quantities like the mass \(m(a)\) and wavefunction renormalization \(Z(a)\) must be adjusted to ensure they remain finite as \(a \to 0\). The mass gap \(m(a)\) is expected to scale with \(a\) as:
\begin{equation}
m(a) = Z(a) m_{\text{phys}} + \mathcal{O}(a),
\end{equation}
where \(m_{\text{phys}}\) is the physical mass in the continuum theory, and \(Z(a)\) is a renormalization factor.

\subsection{Continuum Mass Gap}

The continuum mass gap is defined as the lowest non-zero eigenvalue of the Hamiltonian in the continuum theory, obtained by taking the limit:
\begin{equation}
m_{\text{gap}} = \lim_{a \to 0} m(a).
\end{equation}



\section{Independence of the Regularization Scheme}

\subsection{Universality}

In quantum field theory, the continuum limit is expected to be independent of the specific regularization scheme used (e.g., lattice spacing, entropic regularization). This property, known as universality, implies that physical observables (like the mass gap) depend only on the long-distance behavior of the theory, not on the details of the short-distance regularization.

\subsection{Proof of Independence}

To prove that the mass gap is independent of the regularization scheme, we show that the continuum limit of the theory is governed by the same fixed point in the RG flow, regardless of the regularization scheme.

Consider two different regularization schemes, \(\mathcal{R}_1\) and \(\mathcal{R}_2\), with corresponding path integral measures \(\mathcal{D}U_1 e^{-S_{\mathcal{R}_1}[U]}\) and \(\mathcal{D}U_2 e^{-S_{\mathcal{R}_2}[U]}\). As long as both schemes preserve gauge invariance, locality, and renormalizability, the RG flows of \(g(a)\) for both schemes will converge to the same fixed point \(g^*\).

For the entropic regularization scheme, the total action is \(S[U] = S_W[U] + S_{\text{entropy}}[U]\), where \(S_W[U]\) is the Wilson action and \(S_{\text{entropy}}[U]\) is the entropic regularization term. In the continuum limit, the contribution of the entropic term vanishes as \(\alpha \to 0\) (or as \(a \to 0\) for fixed \(\alpha\)), leading to the same continuum theory as with standard Wilson regularization.



\section{Persistence of the Mass Gap}

\subsection{Mass Gap Calculation}

On the lattice, the mass gap can be computed from the exponential decay of correlation functions, such as the two-point function:
\begin{equation}
G(r; a) \sim e^{-m(a)r},
\end{equation}
where \(r\) is the distance between lattice sites, and \(m(a)\) is the lattice mass gap.

\subsection{Persistence of the Mass Gap}

To show that the mass gap persists in the continuum limit, we demonstrate that \(m(a)\) does not vanish as \(a \to 0\). Non-perturbative effects, such as confinement in Yang-Mills theory, ensure that \(m(a)\) approaches a finite, non-zero value \(m_{\text{gap}}\) as \(a \to 0\). Since the existence of a mass gap is a non-perturbative feature tied to the long-distance behavior of the theory, it remains finite and independent of the regularization scheme in the continuum limit.

\section{Implications}

We have demonstrated the existence of a well-defined continuum limit as the lattice spacing \(a \to 0\) for entropic lattice gauge theory. The mass gap persists in this limit and is independent of the regularization scheme, confirming that it is a genuine feature of the continuum theory and not an artifact of the lattice discretization or entropic regularization.
