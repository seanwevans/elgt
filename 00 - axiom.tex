\section*{Introduction}

We present a complete axiomatic formulation of entropic lattice gauge theory 
and prove the existence and uniqueness of the lattice path integral measure with 
entropic regularization. This framework allows for the systematic exploration 
of non-perturbative effects in gauge theories, such as confinement and the mass 
gap.

\section{Axioms}

\subsection{Discrete Spacetime Lattice}  
Spacetime is represented as a discrete, finite hypercubic lattice with lattice spacing \(a\), consisting of sites indexed by integer coordinates \(x\). The lattice size is \(N^4\) for a four-dimensional lattice with periodic boundary conditions.

\subsection{Gauge Fields as Link Variables}  
The gauge field is represented by link variables \(U_\mu(x) \in G\), where \(G\) is the gauge group (e.g., SU(N)). The link variables are associated with directed edges (links) between nearest-neighbor sites \(x\) and \(x + \hat{\mu}\), where \(\hat{\mu}\) denotes a unit vector in the \(\mu\)-th direction.

\subsection{Gauge Invariance}  
The theory is invariant under local gauge transformations. A gauge transformation \(\Omega(x) \in G\) acts on the link variables as:
\begin{equation}
U_\mu(x) \rightarrow U_\mu'(x) = \Omega(x) U_\mu(x) \Omega^\dagger(x + \hat{\mu}).
\end{equation}

\subsection{Wilson Action}  
The gauge field action is given by the Wilson action:
\begin{equation}
S_W[U] = \frac{\beta}{2} \sum_{x, \mu < \nu} \text{Tr}\left(1 - U_{\mu\nu}(x)\right),
\end{equation}
where \(U_{\mu\nu}(x)\) is the plaquette variable defined by:
\begin{equation}
U_{\mu\nu}(x) = U_\mu(x) U_\nu(x + \hat{\mu}) U_\mu^\dagger(x + \hat{\nu}) U_\nu^\dagger(x).
\end{equation}

\subsection{Entropic Regularization}  
The entropic regularization introduces an additional term in the action, penalizing high-entropy configurations:
\begin{equation}
S_{\text{entropy}}[U] = \alpha \sum_{x, \mu < \nu} \text{Tr}\left(U_{\mu\nu}(x) U_{\mu\nu}^\dagger(x)\right),
\end{equation}
where \(\alpha\) is a positive parameter controlling the strength of the entropic regularization.

\subsection{The Lattice Path Integral}  
The path integral is defined as the sum over all possible configurations of the link variables, weighted by the exponential of the total action:
\begin{equation}
Z = \int \mathcal{D}U \, e^{-S_W[U] - S_{\text{entropy}}[U]/S_{\text{max}}},
\end{equation}
where \(\mathcal{D}U\) denotes the Haar measure over the gauge group for each link variable, and \(S_{\text{max}}\) normalizes the entropic term.



\section{Existence and Uniqueness of the Lattice Path Integral Measure}

To prove the existence and uniqueness of the lattice path integral measure with 
entropic regularization, we proceed by demonstrating:

\subsection{Existence of the Measure}

\subsubsection{Integrability and Finite Action}
The Wilson action \(S_W[U]\) and the entropy functional \(S_{\text{entropy}}[U]\) are both gauge-invariant and finite for any configuration of the link variables \(U_\mu(x)\). The Wilson action \(S_W[U]\) is bounded below by zero since \(\text{Tr}(1 - U_{\mu\nu}(x))\) is positive semi-definite. The entropy functional \(S_{\text{entropy}}[U]\) is also bounded below by zero since it is defined as a sum of positive semi-definite terms \(\text{Tr}(U_{\mu\nu}(x) U_{\mu\nu}^\dagger(x))\).

\subsubsection{Measure Definition}
The Haar measure \(\mathcal{D}U\) for each link variable is finite and well-defined because the gauge group \(G\) is compact (e.g., SU(N)). The integrand \(e^{-S_W[U] - S_{\text{entropy}}[U]/S_{\text{max}}}\) is bounded between 0 and 1, making the path integral convergent.

\subsubsection{Proof of Existence}
Since both actions \(S_W[U]\) and \(S_{\text{entropy}}[U]\) are finite and non-negative for all \(U\), and the Haar measure \(\mathcal{D}U\) is finite, the exponential factors are bounded. Therefore, the path integral \(Z\) is a well-defined finite integral over a compact space, ensuring the existence of the measure.

\subsection{Uniqueness of the Measure}

\subsubsection{Gauge Invariance}
The measure \(\mathcal{D}U \, e^{-S_W[U] - S_{\text{entropy}}[U]/S_{\text{max}}}\) is gauge-invariant by construction because both \(S_W[U]\) and \(S_{\text{entropy}}[U]\) are gauge-invariant.

\subsubsection{Uniqueness Under Discretization}
Any other measure that respects the gauge invariance and the lattice discretization must be equivalent to \(\mathcal{D}U\) due to the uniqueness of the Haar measure for compact groups. If another measure \(\mathcal{D}U' \, e^{-S'[U]}\) respects the same axioms, it must be gauge-invariant and integrate the same observables as \(\mathcal{D}U \, e^{-S_W[U] - S_{\text{entropy}}[U]/S_{\text{max}}}\). Given the compactness of \(G\) and the bounded nature of the action terms, such a measure must coincide with the original measure up to a normalization constant, which does not affect the computation of expectation values.

\subsubsection{Proof of Uniqueness}
Since any valid measure must be gauge-invariant and respect the lattice structure, and since the Haar measure on a compact group is unique, the path integral measure \(\mathcal{D}U \, e^{-S_W[U] - S_{\text{entropy}}[U]/S_{\text{max}}}\) is unique up to normalization.



\section{Implications}

The axiomatic formulation of the entropic lattice gauge theory defines a gauge-invariant and well-posed framework for studying non-perturbative effects. We have demonstrated the existence and uniqueness of the lattice path integral measure with entropic regularization, ensuring that the formulation is mathematically consistent and suitable for further non-perturbative analysis.
