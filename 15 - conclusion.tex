\section{Conclusion}

In this work, we have developed a rigorous framework for constructing the continuum theory from the lattice formulation using constructive field theory techniques, while ensuring the preservation of a non-zero mass gap. Our approach involved several key steps and methodologies:

Lattice Formulation and Constructive Techniques: We began by defining the lattice field theory for scalar and gauge fields, utilizing the Wilson action and other discretized forms of the theory. By employing constructive field theory methods, particularly reflection positivity and the Osterwalder-Schrader axioms, we ensured that the quantum field theory is well-defined and physically meaningful in the continuum limit.

Error Estimates and Convergence Proofs: We provided rigorous error estimates for all approximations used in the construction process, including series expansions in lattice perturbation theory, numerical lattice simulations, and continuum limit approximations. These estimates allowed us to control the errors systematically and prove the convergence of series expansions and numerical methods, ensuring that they approach the exact results as the lattice spacing decreases or the number of terms increases.

Renormalization and Mass Gap Preservation: By analyzing the renormalization group flow and choosing appropriate renormalization conditions, we demonstrated that the renormalized quantities remain finite and well-defined as the lattice spacing a approaches zero. Furthermore, we proved that the mass gap, defined as the energy difference between the ground state and the first excited state, remains positive in the continuum limit. This crucial result shows that the mass gap is a stable feature of the theory, preserved under the renormalization and scaling processes.

Spectral Analysis and Continuum Limit Construction: Using advanced functional analysis techniques, we studied the spectrum of the Hamiltonian and the transfer matrix, proving the existence of a spectral gap above the ground state energy. This spectral gap corresponds to the mass gap, confirming that the lattice formulation accurately captures the essential non-perturbative dynamics of the continuum theory.

Overall, our work provides a comprehensive and mathematically rigorous foundation for understanding how lattice field theories can be used to construct continuum quantum field theories while maintaining key physical features like the mass gap. This approach not only reinforces the validity of lattice formulations as a tool for non-perturbative studies but also contributes to the broader understanding of quantum field theory in both finite and infinite-dimensional settings.

Future research could extend these techniques to more complex theories, such as those involving fermions or supersymmetry, and explore further connections with other areas of mathematical physics, such as conformal field theory and topological quantum field theory.

\end{document}