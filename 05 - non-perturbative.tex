\section{Non-Perturbative Completeness}

We demonstrate that the entropic lattice approach captures all relevant non-
perturbative effects and prove that no additional non-perturbative phenomena 
could alter the conclusion about the mass gap. This involves showing that the 
entropic regularization does not exclude significant configurations and respects 
all non-perturbative phenomena.



\section{Path Integral Formulation and Completeness}

The path integral formulation on the lattice with entropic regularization is given by:
\begin{equation}
Z = \int \mathcal{D}U \, e^{-S_W[U] - S_{\text{entropy}}[U]/S_{\text{max}}},
\end{equation}
where \( \mathcal{D}U \) is the Haar measure over the gauge group for all link variables \( U_\mu(x) \), \( S_W[U] \) is the Wilson action, and \( S_{\text{entropy}}[U] \) is the entropic regularization term.

\subsection{Inclusion of All Configurations}

The path integral sums over all possible configurations of the gauge field. The measure \( \mathcal{D}U \) is complete, integrating over every link variable \( U_\mu(x) \) associated with each lattice link. Thus, all configurations, including non-perturbative effects like instantons, monopoles, and other topological structures, are included.

\subsection{Weighting by Action}

Configurations are weighted by the exponential of the action, \( e^{-S_W[U] - S_{\text{entropy}}[U]/S_{\text{max}}} \). The Wilson action \( S_W[U] \) captures the standard gauge interactions, while \( S_{\text{entropy}}[U] \) ensures the inclusion of entropy-based regularization without excluding relevant physical configurations.



\section{Entropic Regularization and Non-Perturbative Phenomena}

The entropic regularization term:
\begin{equation}
S_{\text{entropy}}[U] = \alpha \sum_{x, \mu < \nu} \text{Tr} \left(U_{\mu\nu}(x) U_{\mu\nu}^\dagger(x)\right),
\end{equation}
penalizes high-entropy configurations but does not suppress significant non-perturbative phenomena.

\subsection{Gauge Invariance and Topological Structures}

The entropic term is gauge-invariant. It depends on the field strength through the plaquette variables \( U_{\mu\nu}(x) \). Because it does not depend on local gauge choices, it preserves topological features such as instantons.

\subsection{Instantons and Topological Charge}

Instantons contribute non-trivially to the gauge field configurations and are characterized by a topological charge \( Q \):
\begin{equation}
Q = \frac{1}{32\pi^2} \int d^4x \, \epsilon^{\mu\nu\rho\sigma} \text{Tr}(F_{\mu\nu} F_{\rho\sigma}).
\end{equation}
The entropic regularization, which penalizes configurations based on the magnitude of the field strength, does not suppress instantons. It respects the topological sector of each configuration because instantons are local minima of the action, not characterized by high entropy.

\subsection{Vortex and Confinement Phenomena}

Confinement is a critical non-perturbative effect in gauge theory, often described in terms of vortex configurations or Wilson loops' area law behavior. The entropic regularization affects the weighting of these configurations but does not exclude them. The mass gap associated with confinement, therefore, remains robust under the entropic lattice formulation.



\section{Topological Quantum Field Theory and Completeness of Non-Perturbative Effects}

In topological quantum field theory (TQFT), all physically distinct gauge configurations are characterized by their topological sector, which corresponds to different classes of gauge transformations.

\subsection{Completeness of Topological Sectors}

The lattice formulation with the entropic regularization respects the decomposition into topological sectors. Each sector is integrated over independently in the path integral, ensuring that all possible configurations (up to gauge equivalence) contribute to the partition function \( Z \).

\subsection{Functional Integration and No Missing Phenomena}

Since the entropic approach maintains the integration over all field configurations and respects gauge invariance and topology, no other non-perturbative phenomena are left out. This completeness implies that any calculation of observables, including the mass gap, includes contributions from all relevant field configurations.



\section{Implications}

The entropic lattice approach captures all relevant non-perturbative effects, including instantons, monopoles, vortices, and topological effects, as it integrates over all configurations respecting gauge invariance. It does not exclude any significant configurations, and the completeness of the path integral over all topological sectors ensures that the mass gap result is comprehensive.
