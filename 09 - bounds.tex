\section{Bounds on Correlation Functions}

We establish upper and lower bounds on the correlation functions and prove 
inequalities that constrain their behavior, ensuring a non-zero mass gap. This 
replaces numerical evidence with a rigorous mathematical framework.



\section{Definition of the Mass Gap}

\subsection{Mass Gap and Two-Point Correlation Function}

The mass gap \(m_{\text{gap}}\) is the lowest non-zero eigenvalue of the Hamiltonian in the theory or, equivalently, the inverse of the correlation length \(\xi\). For a scalar field \(\phi(x)\), the two-point correlation function \(G(x - y)\) is defined by:
\begin{equation}
G(x - y) = \langle \phi(x) \phi(y) \rangle.
\end{equation}
In a theory with a mass gap, this function should decay exponentially at large distances:
\begin{equation}
G(x - y) \sim e^{-m_{\text{gap}} |x - y|} \quad \text{as} \quad |x - y| \to \infty.
\end{equation}



\section{Upper and Lower Bounds on Correlation Functions}

\subsection{Upper Bound on Correlation Functions}

Using the Cauchy-Schwarz inequality, we have:
\begin{equation}
G(x) = \langle \phi(0) \phi(x) \rangle \leq \sqrt{\langle \phi(0)^2 \rangle \langle \phi(x)^2 \rangle}.
\end{equation}
Since \(\langle \phi(x)^2 \rangle = G(0)\), we can write:
\begin{equation}
G(x) \leq G(0).
\end{equation}

Using the spectral representation:
\begin{equation}
G(x) = \int_0^\infty d\mu^2 \, \rho(\mu^2) e^{-\mu |x|},
\end{equation}
where \(\rho(\mu^2) \geq 0\) is the spectral density, and for any \(\mu > 0\):
\begin{equation}
G(x) \leq e^{-\mu |x|} \int_0^\infty d\mu^2 \, \rho(\mu^2) = e^{-\mu |x|}.
\end{equation}
Choosing \(\mu = m_{\text{gap}}\), we find:
\begin{equation}
G(x) \leq e^{-m_{\text{gap}} |x|}.
\end{equation}

\subsection{Lower Bound on Correlation Functions}

Consider a test function \(f(x)\) with compact support and write the functional form:
\begin{equation}
F[f] = \int d^dx \, G(x) f(x).
\end{equation}
By positivity of the inner product, we have:
\begin{equation}
F[f] \geq 0.
\end{equation}

Choose \(f(x)\) such that it peaks around some large \(|x|\) and decays quickly. Then, the integral mainly gets contributions from regions where \(f(x)\) is non-zero. For large \(|x|\), the behavior of \(G(x)\) dictates that:
\begin{equation}
F[f] \sim e^{-m_{\text{gap}} |x|}.
\end{equation}
Hence, the non-negativity of \(F[f]\) implies:
\begin{equation}
G(x) \geq c e^{-m_{\text{gap}} |x|}
\end{equation}
for some positive constant \(c > 0\).



\section{Inequalities for Spectral Representation}

\subsection{Spectral Representation and Bounds}

The spectral density \(\rho(\mu^2)\) is non-negative and normalized:
\begin{equation}
\int_0^\infty d\mu^2 \, \rho(\mu^2) = 1.
\end{equation}
Given that \(\rho(\mu^2) = 0\) for \(\mu < m_{\text{gap}}\), the spectral representation can be bounded from below by:
\begin{equation}
G(x) = \int_{m_{\text{gap}}^2}^\infty d\mu^2 \, \rho(\mu^2) e^{-\mu |x|}.
\end{equation}
Since \(\rho(\mu^2) \geq 0\), this representation gives:
\begin{equation}
G(x) \geq e^{-m_{\text{gap}} |x|} \int_{m_{\text{gap}}^2}^\infty d\mu^2 \, \rho(\mu^2).
\end{equation}

As \(\int_{m_{\text{gap}}^2}^\infty d\mu^2 \, \rho(\mu^2) = 1\), we find:
\begin{equation}
G(x) \geq e^{-m_{\text{gap}} |x|}.
\end{equation}

Combining upper and lower bounds, we have:
\begin{equation}
e^{-m_{\text{gap}} |x|} \leq G(x) \leq e^{-m_{\text{gap}} |x|}.
\end{equation}



\section{Functional Inequalities}

\subsection{Correlation Function Inequality}

Applying Jensen's inequality to the spectral representation, we see that:
\begin{equation}
\log G(x) = \log \left( \int_{m_{\text{gap}}^2}^\infty d\mu^2 \, \rho(\mu^2) e^{-\mu |x|} \right) \leq \int_{m_{\text{gap}}^2}^\infty d\mu^2 \, \rho(\mu^2) \log(e^{-\mu |x|}).
\end{equation}
Simplifying this:
\begin{equation}
\log G(x) \leq -m_{\text{gap}} |x|.
\end{equation}
Exponentiating both sides, we find:
\begin{equation}
G(x) \leq e^{-m_{\text{gap}} |x|}.
\end{equation}



\section{Implications}

We have established an upper and lower bounds on the correlation functions, demonstrating that they decay exponentially at large distances with a rate defined by the mass gap \(m_{\text{gap}}\). These bounds ensure that the mass gap is non-zero, replacing any reliance on numerical evidence with rigorous mathematical inequalities.
